
% -------------------------------------------------------
%  English Abstract
% -------------------------------------------------------


\pagestyle{empty}

\begin{latin}

\begin{center}
\textbf{Abstract}
\end{center}
\baselineskip=.8\baselineskip

Effective tracking and re-identification of persons is essential for analyzing team sport videos. However, this task is challenging due to the nonlinear motion of players, the similarity in appearance of players from the same team, the distance of the camera from the persons on the pitch, and frequent occlusions. Therefore, the ability to extract meaningful embeddings to represent persons is crucial in developing an effective tracking and re-identification system. In team sports, there is other information that can be used for re-identification of persons, such as team affiliation, role information, and jersey number. However, existing methods usually suffer from two problems: first, training separate networks for each of those tasks comes with high computational costs, and second, heavy occlusions and similar appearances in sports videos limit the solutions for these tasks.  In this research, a multi-purpose part-based person representation method, called PRTreID, is proposed that performs three tasks of Role Classification, Team Affiliation, and Re-Identification, simultaneously. In contrast to available literature, a single network is trained with multi-task supervision to solve all three tasks jointly, which is computationally effective. The proposed method is part-based, leveraging body part-based information for each person, which can be significantly helpful in occlusion scenarios. The multi-task learning leads to richer and more discriminative representations, as demonstrated by both quantitative and qualitative results. Furthermore, the proposed PRTreID model is integrated with a re-identification based tracking method and a part-based post-processing module to handle long-term tracking is proposed. With the powerful re-identification model, the resulting tracking method, named PRT-Track, is capable of effectively re-identifying persons even in challenging scenarios, outperforming all recent tracking methods on the challenging SoccerNet-Tracking dataset. The proposed method improves the best existing method in terms of HOTA and AssA by 1.57 and 2.53 percentage points, respectively.



\bigskip\noindent\textbf{Keywords}:
Computer Vision, Deep Learning, Team Sports Videos, Re-Identification, Multi-Object Tracking, Part-based Re-Identification, Team Affiliation, Multi-task Learning, Representation Learning
\end{latin}

\newpage
