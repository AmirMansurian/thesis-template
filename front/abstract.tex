
% -------------------------------------------------------
%  Abstract
% -------------------------------------------------------

\vspace*{-60pt}
\enlargethispage{90pt}

\شروع{وسط‌چین}
\مهم{چکیده}

\پایان{وسط‌چین}
\vspace{-1cm}

\بدون‌تورفتگی


ردیابی و ‌‌بازشناسایی موثر افراد برای تجزیه و تحلیل ویدیوهای ورزش تیمی امری ضروری است. این کار اما به دلیل حرکت غیرخطی بازیکنان در مقایسه با عابر پیاده، شباهت ظاهری بازیکنان یک تیم، فاصله دوربین از افراد حاضر در زمین و انسدادهای مکرر، کاملا چالش برانگیز است. بنابراین، توانایی استخراج بازنمایی‌های معنادار برای نمایش افراد در توسعه یک سیستم ردیابی و بازشناسایی موثر بسیار مهم است. در ورزش های تیمی اما، اطلاعات دیگری وجود دارد که می تواند در بازشناسایی افراد استفاده شود، مانند وابستگی‌تیم، اطلاعات نقش هر فرد و شماره پیراهن. با ظهور شبکه های عصبی پیچشی ژرف و پیشرفت آن‌ها در وظایف ورزشی، در حال حاضر شبکه‌های خوبی با دقت بالا برای وظایف گفته شده وجود دارد. با این حال، روش‌های موجود معمولاً از دو مشکل رنج می‌برند: اول، آموزش شبکه‌های مجزا برای هر یک از آن وظایف با هزینه‌های محاسباتی بالایی همراه است، و دوم، انسدادهای سنگین و ظاهر مشابه در ویدیوهای ورزشی، که دقت راه‌حل‌های موجود برای این کارها را محدود می‌کند. در این پژوهش، یک روش بازنمایی فرد پاره-محور چند منظوره به نام $PRTreID$ پیشنهاد شده است که سه وظیفه دسته‌بندی نقش، وابستگی‌تیم و بازشناسایی را به طور همزمان انجام می‌دهد. برخلاف کارهای موجود، یک شبکه واحد با نظارت چند وظیفه‌ای برای حل هر سه کار به طور مشترک آموزش داده می شود که از نظر محاسباتی موثر است. روش پیشنهادی پاره-محور است و از اطلاعات مبتنی بر قسمت‌های مختلف بدن برای هر فرد بهره می‌برد که می‌تواند به طور قابل توجهی در فرنامه‌های انسداد مفید باشد. همانطور که توسط نتایج کمی و کیفی نشان داده شده‌است، یادگیری چند وظیفه‌ای منجر به بازنمایی های غنی تر و تمایز دهنده تر می‌شود. علاوه بر این، مدل $PRTreID$ پیشنهادی با یک روش ردیابی مبتنی بر بازشناسایی یکپارچه شده و یک الگوریتم پس‌پردازش پاره-محور برای مدیریت ردیابی بلند‌مدت پیشنهاد شده است. با مدل چندمنظوره بازشناسایی پیشنهادی، روش ردیابی به دست آمده، $PRT-Track$، قادر است به طور موثر افراد را حتی در فرنامه‌های چالش برانگیز ردیابی کند، و از همه روش های ردیابی اخیر در مجموعه‌داده چالش برانگیز $SoccerNet-Tracking$ بهتر عمل‌کند. روش پیشنهادی معیارهای $HOTA$ و $AssA$ را نسبت به بهترین روش موجود، به ترتیب، به مقدار $1.57$ و $2.53$ درصد بهبود می‌دهد.



\پرش‌بلند
\بدون‌تورفتگی \مهم{کلیدواژه‌ها}: 
بینایی کامپیوتر، یادگیری ژرف، ویدیوهای ورزش تیمی، بازشناسایی، ردیابی چندشیء، بازشناسایی پاره-محور، وابستگی‌تیم، یادگیری چندوظیفه‌ای، یادگیری بازنمایی 
\صفحه‌جدید
